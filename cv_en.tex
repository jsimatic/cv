\documentclass[11pt, a4paper]{moderncv}
\moderncvtheme[orange]{classic}

\usepackage[utf8]{inputenc}

% Adjust the page margins
\usepackage[scale=0.85]{geometry}
\recomputelengths
\newcommand{\ALPS}{Architectural tools for ultra-Low Power (event-driven) Systems}
%\sethintscolumntowidth{November 2014}

% Personal data
\firstname{Jean}
\familyname{Simatic}
\address{30 rue Félix Esclangon}{38000 Grenoble, France}
\mobile{+33 6 28 13 37 12}
\email{jean@simatic.org}
%\photo[64pt]{}
%\nopagenumbers{}
\title{Asynchronous microelectronics PhD candidate}
\begin{document}
\maketitle

\vspace{-.75cm}

\section{Education}
\cventry{2014 --- 2017}{PhD Thesis}{TIMA Laboratory}{Grenoble, France}{}{%
  Design flow for ultra-low power: non-uniform sampling and asynchronous circuits}
\cventry{2013 --- 2014}{Master}{ENSTA}{%
  Robotics and Embedded Systems}{}{%
  Multiprocessors on chip, embedded software, robotics, mecatronics}
\cventry{2013 --- 2014}{Master}{UPMC}{%
  Electronic Systems and Computer Systems}{}{%
  Mixed and analog circuit design, noise, design for test, MEMS}
\cventry{2010 --- 2013}{Engineering}{{\'E}cole polytechnique}{Electrical
  Engineering major}{}{Digital circuit design, processor architecture,
  semi-conductors, optoelectronics,\\ network (Internet), statistics}

\section{Experience}
\cventry{2017 --- Auj.}{Startup incubation and executive direction}{Hawai.tech}%
        {Grenoble, France}{}{%
          \begin{itemize}
          \item Design of a modular architecture for low-power probabilistic circuits.
          \item Commercial and financial development.
          \end{itemize}
        }
\cventry{2014\\6 months}{Intern}{Tiempo}%
        {Montbonnot Saint Martin, France}{}{%
          \begin{itemize}
          \item Design of a verification tool for standard cells Verilog modules
          \item Evaluation of a commercial fault simulator on QDI circuits
          \end{itemize}
        }
\cventry{2013\\4 months}{Intern}{Asynchronous Research Center}%
        {Portland (Oregon), USA}{}{%
          \begin{itemize}
          \item Design of asynchronous IP for merge sort
          \item Contribution to the development of a CAD tool (ARCWelder)
          \end{itemize}
        }
\cventry{2012\\1 months}{Intern}{EADS Astrium, Digital electronic department}%
        {{\'E}lancourt, France}{}{%
          \begin{itemize}
          \item Development of a test environment for a spatial CAN bus
          \end{itemize}
        }
\cventry{2011 --- 2012\\1 year}{Electronic designer and treasurer}{Robotic
  club at {\'E}cole polytechnique}%
        {}{}{%
          Participation to French Robotic Cup
          \begin{itemize}
          \item Design and fabrication of the electronic boards of the robot
          \item Financial management of the project
          \end{itemize}
        }

\section{Skills}
\cvline{EDA Tools}{ModelSim, Design Compiler, Quartus}
\cvline{Programming}{VHDL, Verilog, SystemC, Spice, Python, Java, GNU Make, C/C++, PHP, Caml}
\cvline{OS}{Linux (Ubuntu, ArchLinux), Windows, RTEMS}
\cvline{Misc}{Git, Subversion, \LaTeX, Scilab, Matlab, Eclipse}

\section{Languages}
\cvcomputer{French}{Native language}{German}{Conversational}
\cvcomputer{English}{Fluent}{Portuguese}{Conversational}

\section{Interests}
\cvline{Music}{Viola and trumpet player in orchestras and marching bands. Classical and Jazz amateur.}
\cvline{Sports}{Badminton, hiking and cross-country skiing.}
\cvline{Pyrotechnics}{C4/T2 trained firework firer.}

\closesection{}
\pagebreak

\sethintscolumntowidth{Co-advisor}

% Academic
\section{PhD Thesis}
\cvline{Title}{Design flow for ultra-low power: Non-uniform sampling and
  asynchronous circuits}
\cvline{Advisor}{Laurent Fesquet}
\cvline{Co-advisor}{Rodrigo Possamai Bastos}
\cvline{Abstract}{%
  The Internet of Things requires developing ultra-low power platforms embedding
  actuators, sensors, and signal processors. Event-driven sampling and circuitry
  allow reducing the amount of sampled data, the system activity, and therefore
  the power consumption. For helping designers in rapidly developing
  event-driven ultra-low power platforms, we have devised a complete framework
  named ALPS: \ALPS{}. ALPS framework will allow to choose and simulate a
  signal-specific sampling scheme, and to synthesize a dedicated event-driven
  circuit for processing the non-uniformly sampled data. The estimated power
  reduction factor for a the event-driven version of a filter systems is from 3
  to 30 depending on the input signal activity.
}

\renewcommand{\refname}{Publications}
\nocite{Simatic17a, Skaf17, Qaisar17, El-Hadbi17, Simatic16a, Simatic16, Simatic15}
\bibliographystyle{IEEEtran}
\bibliography{publications}
\end{document}
