\documentclass[11pt]{moderncv}
\moderncvtheme[orange]{classic}

\usepackage[utf8]{inputenc}

% Adjust the page margins
\usepackage[scale=0.88]{geometry}
\recomputelengths
\newcommand{\ALPS}{Architectural tools for ultra-Low Power (event-driven) Systems}
%\sethintscolumntowidth{November 2014}

% Personal data
\firstname{Jean}
\familyname{Simatic}
\address{30 rue Félix Esclangon}{38000 Grenoble, France}
\mobile{+33 6 28 13 37 12}
\email{jean@simatic.org}
%\photo[64pt]{}
%\nopagenumbers{}
\title{Directeur technique \& cofondateur startup Hawai.tech}
\begin{document}
\maketitle

\vspace{-.75cm}
\section{Expérience}
\cventry{2021 --- Auj.}{Direction technique de startup}{Hawai.tech}%
        {Grenoble, France}{}{%
          \begin{itemize}
          \item Développement et maintien de la roadmap technologique. Veille technologique.
          \item Management d'équipes technique (5 pers.) et coordination de projets.
          \item À temps partiel, développement de circuits électioniques et gestion IT.
          \end{itemize}
        }
\cventry{2018 --- 2021}{Incubation et présidence d'une startup}{Hawai.tech}%
        {Grenoble, France}{}{%
          \begin{itemize}
          \item Montage business, juridique et financier et développement de 0 à 10 salariés.
          \item Levée 340 k€ (Business Angels \& corporate). Négociation de contrats (Brevet + 100k€ CA).
          \end{itemize}
        }
\cventry{2014\\6 mois}{Stage de fin d'étude}{Tiempo Secure}%
        {Montbonnot Saint Martin, France}{}{%
          \begin{itemize}
          \item Conception d'un outil de vérification pour des modèles
            Verilog de cellules standards.
          \item \'Evaluation d'un outils commercial de simulation de
            fautes sur des circuits QDI.
          \end{itemize}
        }
\cventry{2013\\4 mois}{Stage de recherche}{Asynchronous Research Center}%
        {Portland (Oregon), USA}{}{%
          \begin{itemize}
          \item Conception de composants asynchrones réalisant un tri fusion.
          \item Contribution au développement d'un outil de CAO (ARCWelder).
          \end{itemize}
        }
\cventry{2012\\1 mois}{Stage}{EADS Astrium, Groupe électronique numérique}%
        {Élancourt, France}{}{%
          \begin{itemize}
          \item Développement d'un environnement de test pour un bus CAN spatial.
          \end{itemize}
        }
% \cventry{2011 -- 2012\\1 an}{Coupe de Fance de robotique}{Club de
%   robotique de l'École polytechnique}%
%         {}{}{%
%           \begin{itemize}
%           \item Co-responsable électronique : Conception et réalisation des cartes électroniques du robot
%           \item Trésorier : Gestion financière du projet
%           \end{itemize}
%         }

\section{Formation}
\cventry{2014 --- 2017}{Thèse de doctorat}{Laboratoire TIMA}{Grenoble, France}{}{%
  Flot de conception pour la faible consommation :
  échantillonage non uniforme et circuits asynchrones}
\cventry{2013 --- 2014}{Master}{Université Pierre et Marie Curie}{%
  Systèmes électroniques et systèmes informatiques}{}{%
  Circuits mixtes et analogiques, bruit, conception pour le test, MEMS}
\cventry{2013 --- 2014}{Ingénieur}{ENSTA ParisTech}{%
  Robotique et Systèmes Embarqués}{}{%
  Multiprocesseur sur puce, logiciel embarqué, robotique, mécatronique}
\cventry{2010 --- 2013}{Ingénieur}{École polytechnique}{%
  Electrical Engineering}{}{%
  Circuits numériques ASIC et FPGA, architecture des processeurs,
  semi-conducteurs,\\optoélectronique, réseau, statistiques}
% \cventry{2008 --- 2010}{Classe préparatoires}{Lycée Janson de Sailly}{%
%   Maths Physique option Informatique}{}{}
% \cventry{2008}{Baccalauréat}{Lycée Jules Verne}{%
%   Limours, France}{Mention Très bien}{}

\section{Compétences}

\cvline{Savoir-êtres}{Écoute, initiative, polyvalence, efficacité}
\cvline{Outils CAO}{ModelSim, Vivado, Design Compiler, Quartus, CatapultC}
\begingroup
\sethintscolumntowidth{Programmation}
\cvline{Programmation}{VHDL, SystemVerilog, SystemC, Python, Spice, GNU Make, C/C++, Java}
\endgroup
\cvline{OS}{Linux (Ubuntu, ArchLinux, CentOS), Windows}
\cvline{Divers}{Git, AWS EC2/EFS, Wireguard, Docker, Subversion, \LaTeX}

\section{Langues}
\cvcomputer{Français}{Langue martenelle}{Allemand}{Lu, écrit, parlé}
\cvcomputer{Anglais}{Courant}{Portuguais}{Lu, écrit, parlé}

\section{Intérêts}
\cvline{Musique}{Altiste et trompettiste en orchestre et en fanfare. Amateur de
  musique classique et jazz.}
\cvline{Sports}{Badminton, randonnée et ski de fond.}
\cvline{Pyrotechnie}{Artificier formé C4/T2.}

\closesection{}
\pagebreak

% Academic
\section{Thèse de doctorat}
\cvline{Titre}{Flot de conception pour l'ultra-faible consommation : %
  échantillonage non uniforme et circuits asynchrone}
\cvline{Encadrement}{Laurent Fesquet (Directeur), Rodrigo Possamai
  Bastos (Co-encadrant)}
\cvline{Résumé}{%
  L'internet des objets nécessite le développement de plateformes peu
  consommantes embarquant actuateurs, capteurs et traîtement du
  signal. L'échantillonage et les circuits basés sur les évènements
  permettent de réduire la quantité de données échantillées,
  l'activité du circuit et donc la consommation. Pour aider les
  concepteurs dans le développement rapide de plateformes ultra-faible
  consommation et basées sur les évènements, cette thèse présente un
  flot complet ALPS: \ALPS{}. Le framework ALPS permet de choisir et
  simuler un schéma d'échantillonage spécifique au signal visé et de
  synthétiser un circuit asynchrone dédié pour traîter les données
  échantillées non uniformément.}
\cvline{Mots clés}{échantillonage non uniforme, circuits asynchrones,
  synthèse de haut niveau}

\section{Enseignement}
\cventry{2015---2017}{Monitorat}{Phelma}{Grenoble, France}{128h Eq. TD}{%
          \begin{itemize}
          \item Travaux dirigés niveau M1 : OS et Réseau (64h), Conception VHDL (32h)
          \item Tutorat de projets étudiants niveau L3 (12h)
          \item Ateliers conception Nano@School (18h) : Sensibiliser des lycéens
            aux outils de la microélectronique.
          \end{itemize}
        }


\section{Publications}
\begingroup
\renewcommand{\section}[1]{}%
%\renewcommand{\chapter}[2]{}% for other classes
\sethintscolumntowidth{[1]}
\renewcommand{\refname}{}
\nocite{Simatic17a, Skaf17, Qaisar17, El-Hadbi17, Simatic16a, Simatic16, Simatic15}
\bibliographystyle{IEEEtran}
\subsection{Conférences internationales à comité de relecture}
\bibliography{publications}
\endgroup
\end{document}
